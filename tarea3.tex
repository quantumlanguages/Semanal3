\documentclass{article}

% formato
\usepackage[margin = 1.5cm, letterpaper]{geometry}
\usepackage[utf8]{inputenc}

% autómatas
\usepackage{tikz}
\usetikzlibrary{automata, positioning}

%formato ecuaciones
\usepackage{amsmath}

% símbolos
\usepackage{amssymb}

% manejo de tablas
\usepackage{float}

\begin{document}
    \title{
        Lenguajes de Programación \\
        Ejercicio Semanal 3
    }

    \author{
        Sandra del Mar Soto Corderi \\
        Edgar Quiroz Castañeda
    }

    \date{
        22 de agosto del 2019
    }
    
    \maketitle

    \begin{enumerate}
        \item {
            Responde cada inciso a partir de la siguiente expresión $e$
            
            let x = let y = 3 in 2 + y end in\\
            let w = let y = 2 in y * 5 end in\\
            let y = w / x in y + x end + 2\\
            end\\
            end
            
            \begin{enumerate}
            	\item { Llenar la siguiente tabla con base en $e$. El orden de las expresiones let es, por supuesto, el orden de lectura
            		como texto.
            		
            		\begin{table}[H]
            			\centering
            			\begin{tabular}{|l|l|l|l|}
            				\hline
            				let & variable ligada & expresión a ligar & alcance del let \\ \hline
            				1   & x               &                   &                 \\ \hline
            				2   & y               &                   &                 \\ \hline
            				3   & w               &                   &                 \\ \hline
            				4   & y               &                   &                 \\ \hline
            				5   & y               &                   &                 \\ \hline
            			\end{tabular}
            		\end{table}
            	}
            \item {
            Encuentra la representación en la sintaxis abstracta de orden superior de $e$. Está permitido usar sintaxis concreta para operaciones y números.\\
        	
        	}
        	\item{
        	Encuentra una expresión $e'$ que sea $\alpha$-equivalente a $e$ y en donde todas las variables ligadas tengan distinto nombre.\\
        	
        	}
        	\item {
            ¿Cuál es el valor de $e$? Explica como se llego al resultado.
            
            \begin{align*}
                &\texttt{let} \ x = \texttt{let} \ y = 3 \ \texttt{in} \ 2 + y \ \texttt{end} \  \ \texttt{in} \\\
                &\texttt{let} \ w = \texttt{let} \ y = 2 \ \texttt{in} \ y * 5 \ \texttt{end} \  \ \texttt{in} \\\
                &\texttt{let} \ y = w / x \ \texttt{in} \ y + x \ \texttt{end} \  + 2\\
                &\ \texttt{end} \ \\
                &\ \texttt{end} \ \\
                &\rightarrow (eletf) \\
                &\texttt{let} \ x = (2 + y)[y:=3] \ \ \texttt{in} \\
                &\texttt{let} \ w =  (y * 5)[y:=2] \ \texttt{in} \\
                &\texttt{let} \ y = w / x \ \texttt{in} \ y + x \ \texttt{end} \  + 2\\
                &\ \texttt{end} \ \\
                &\ \texttt{end} \ \\
                &= \\
                &\texttt{let} \ x = (2 + 3) \ \ \texttt{in} \\
                &\texttt{let} \ w =  (2 * 5) \ \texttt{in} \\
                &\texttt{let} \ y = w / x \ \texttt{in} \ y + x \ \texttt{end} \  + 2\\
                &\ \texttt{end} \ \\
                &\ \texttt{end} \ \\
                &\rightarrow (esumf), (eprodf), (eleti) \\
                &\texttt{let} \ x = 5 \ \ \texttt{in} \\
                &\texttt{let} \ w =  10 \ \texttt{in} \\
                &\texttt{let} \ y = w / x \ \texttt{in} \ y + x \ \texttt{end} \  + 2\\
                &\ \texttt{end} \ \\
                &\ \texttt{end} \ \\
                &\rightarrow (eletf) \\
                &(\texttt{let} \ y = w / x \ \texttt{in} \ y + x \ \texttt{end} \  + 2)
                [x := 5, w := 10]\\
                &= \\
                &\texttt{let} \ y = (w / x)[x := 5, w := 10] \ \texttt{in} \ 
                (y + x) [x := 5, w := 10] \ \texttt{end} \  + 2 \\
                &\rightarrow (eleti) \\
                &\texttt{let} \ y = 10 / 5 \ \texttt{in} \ y + 5\ \texttt{end} \  + 2 \\
                &\rightarrow (eprodf), (eleti) \\
                &\texttt{let} \ y = 2 \ \texttt{in} \ y + 5\ \texttt{end} \  + 2 \\
                &\rightarrow (eletf) \\
                &(y+5)[y:=2] + 2 \\
                &= \\
                & 2 + 5 + 2 \\
                &\rightarrow (esumf) \\
                & 9
            \end{align*}
        	
        	}
            \end{enumerate}
        }
    	\item {
    	Realiza la siguiente sustitución mostrando la respuesta paso a paso:
    	
        (let  $x = y \ast 4 \ in$ (let $z = x + 3$  in  $y \ast 2$ end) $\ast y $ end) $[y := x + 2]$

        Primero, tenemos que $x \in FV(x + 2)$, por lo que no se puede realizar
        una sustitución textual en el \texttt{let} exterior.

        \[
            (\texttt{let} \ x = y \ast 4 \ \texttt{in} \ 
            (\texttt{let} \ z = x + 3 \ \texttt{in} \ 
            y \ast 2 \ \texttt{end}) \ \ast y \ \texttt{end}) 
            [y := x + 2]
        \]

        Entonces, hay que encontrar una expresión $\alpha$-equivalente que no
        tenga este problema.

        \[
            \equiv_{\alpha}
            (\texttt{let} \ w = y \ast 4 \ \texttt{in} \ 
            (\texttt{let} \ z = w + 3 \ \texttt{in} \ 
            y \ast 2 \ \texttt{end}) \ \ast y \ \texttt{end}) 
            [y := x + 2]
        \]

        Luego, ya es posible realizar la sustitución de forma textual.

        Primero, se aplica recursivamente la sustitución a la asignación y al
        cuerpo del \texttt{let} exterior.

        \[
            =
            \texttt{let} \ w = (y \ast 4)[y := x + 2] \ \texttt{in} \ 
            ((\texttt{let} \ z = w + 3 \ \texttt{in} \ 
            y \ast 2 \ \texttt{end}) \ \ast y) [y := x + 2] \ \texttt{end}
        \]

        En la expresión de asignación ya se puede aplicar directamente la
        sustitución. En el cuerpo del \texttt{let} aún hay que realizarla
        recursivamente. 

        Notemos que en el \texttt{let} interior, se puede realizar la
        sustitución textual sin necesidad de realizar ninguna
        $\alpha$-equivalencia.

        \[
            =
            \texttt{let} \ w = (x + 2) \ast 4 \ \texttt{in} \ 
            (\texttt{let} \ z = w + 3 \ \texttt{in} \ 
            y \ast 2 \ \texttt{end})[y := x + 2] \ \ast y [y := x + 2] \ 
            \texttt{end} 
        \]

        Se aplica la sustitución donde es posible, y se sigue recursivamente
        donde no se puede.

        \[
            =
            \texttt{let} \ w = (x + 2) \ast 4 \ \texttt{in} \ 
            (\texttt{let} \ z = (w + 3) [y := x + 2] \ \texttt{in} \ 
            (y \ast 2 ) [y := x + 2]\ \texttt{end}) \ \ast (x + 2) \ 
            \texttt{end} 
        \]

        Finalmente, se termina de aplicar la sustitución en todas las
        subexpresiones.

        \[
            =
            \texttt{let} \ w = (x + 2) \ast 4 \ \texttt{in} \ 
            (\texttt{let} \ z = (w + 3) \ \texttt{in} \ 
            ((x + 2) \ast 2 ) \ \texttt{end}) \ \ast (x + 2) \ 
            \texttt{end}
        \]
    	
    	}
    \end{enumerate}
\end{document}