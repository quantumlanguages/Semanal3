\documentclass{article}

% formato
\usepackage[margin = 1.5cm, letterpaper]{geometry}
\usepackage[utf8]{inputenc}

% autómatas
\usepackage{tikz}
\usetikzlibrary{automata, positioning}

%formato ecuaciones
\usepackage{amsmath}

% símbolos
\usepackage{amssymb}

% manejo de tablas
\usepackage{float}

\begin{document}
    \title{
        Lenguajes de Programación \\
        Ejercicio Semanal 3
    }

    \author{
        Sandra del Mar Soto Corderi \\
        Edgar Quiroz Castañeda
    }

    \date{
        22 de agosto del 2019
    }
    
    \maketitle

    \begin{enumerate}
        \item {
            Responde cada inciso a partir de la siguiente expresión $e$
            
            let x = let y = 3 in 2 + y end in\\
            let w = let y = 2 in y * 5 end in\\
            let y = w / x in y + x end + 2\\
            end\\
            end
            
            \begin{enumerate}
            	\item { Llenar la siguiente tabla con base en $e$. El orden de las expresiones let es, por supuesto, el orden de lectura
            		como texto.
            		
            		\begin{table}[H]
            			\centering
            			\begin{tabular}{|l|l|l|l|}
            				\hline
            				let & variable ligada & expresión a ligar & alcance del let \\ \hline
            				1   & x               &                   &                 \\ \hline
            				2   & y               &                   &                 \\ \hline
            				3   & w               &                   &                 \\ \hline
            				4   & y               &                   &                 \\ \hline
            				5   & y               &                   &                 \\ \hline
            			\end{tabular}
            		\end{table}
            	}
            \item {
            Encuentra la representación en la sintaxis abstracta de orden superior de $e$. Está permitido usar sintaxis concreta para operaciones y números.\\
        	
        	}
        	\item{
        	Encuentra una expresión $e'$ que sea $\alpha$-equivalente a $e$ y en donde todas las variables ligadas tengan distinto nombre.\\
        	
        	}
        	\item {
        	¿Cuál es el valor de $e$? Explica como se llego al resultado.\\
        	
        	}
            \end{enumerate}
        }
    	\item {
    	Realiza la siguiente sustitución mostrando la respuesta paso a paso:\\
    	
    	(let  $x = y \ast 4 \ in$ (let $z = x + 3$  in  $y \ast 2$ end) $\ast y $ end) $[y := x + 2]$
    	
    	}
    \end{enumerate}
\end{document}